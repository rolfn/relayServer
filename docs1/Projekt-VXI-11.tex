%---------------------------
\listfiles
\documentclass[%
fontsize=11pt
,paper=a4
,twoside
,headings=small
,pointednumbers
,pagesize
]{scrartcl}
\usepackage{hyphsubst}% Trennregeln austauschen
\HyphSubstIfExists{ngerman-x-latest}{%
  \HyphSubstLet{ngerman}{ngerman-x-latest}}{}

\usepackage[utf8]{inputenc}

\usepackage{cmap}
\usepackage[T1]{fontenc}
\usepackage[largesmallcaps]{kpfonts}
\renewcommand{\sfdefault}{cmbr}
\usepackage[scaled]{luximono}
\usepackage{microtype}
\DeclareMicrotypeAlias{jkp}{ppl}

\usepackage[obeyspaces,spaces]{url}
\usepackage[ngerman]{babel}
\usepackage[autostyle=true,babel=once,german=guillemets,maxlevel=3]{csquotes}%
\defineshorthand{"`}{\openautoquote}
\defineshorthand{"'}{\closeautoquote}
\usepackage[%
,style=fiwi,dashed=true,pages=bib,publisher=true,yearatbeginning=false,%
  ibidtracker=false,useprefix=false
,language=auto
,hyperref=auto
,abbreviate=true
,sorting=nyt
,bibencoding=utf8
,block=ragged
,backend=biber
,useprefix=true
,backrefstyle=two
,sortlocale=de_DE
,dateabbrev=false
,datezeros=false
,maxnames=3
,minnames=3
,defernumbers=true
]{biblatex}


%\setkomafont{minisec}{\normalfont\bfseries}
%\setkomafont{disposition}{\normalfont\bfseries}
%\setkomafont{descriptionlabel}{\normalfont\bfseries}

\usepackage[neverdecrease]{paralist}
\makeatletter
\newcommand*\RN@begin@list[1]{%
  \@ovxx=\parindent
  \@ovyy=\parskip
  \@nameuse{#1}%
  \parindent=\@ovxx
  \parskip=\@ovyy
}
\def\enumerate{%
  \RN@begin@list{compactenum}%
}
\def\itemize{%
  \RN@begin@list{compactitem}%
}
\def\description{%
  \RN@begin@list{compactdesc}%
}
\let\enditemize\endcompactitem
\let\endenumerate\endcompactenum
\let\enddescription\endcompactdesc
\pltopsep=\medskipamount
\plitemsep=\medskipamount
\makeatother

\renewcommand*\labelitemii{$\circ$}

\makeatletter
\newcommand\Minisec[1]{\@afterindentfalse \vskip 1.5ex
  {\parindent \z@
    \raggedsection\normalfont\sectfont\nobreak
    \usekomafont{minisec}#1\nobreak}\nobreak%
  \@afterheading
}
\makeatother

\usepackage{shortvrb}
\MakeShortVerb{\|}

\usepackage{xcolor,listings}
\lstset{%
basicstyle=\footnotesize\ttfamily,
identifierstyle={},
keywordstyle=\bfseries,
commentstyle=\itshape,
columns=fixed,
tabsize=2,
frame=single,
extendedchars=true,
showspaces=false,
showstringspaces=false,
breaklines=true,
breakindent=10pt,
backgroundcolor=\color{black!10},
breakautoindent=true,
captionpos=t,
aboveskip=\medskipamount,
belowskip=\medskipamount,
xrightmargin=\fboxsep,
prebreak=\mbox{$\hookleftarrow$},
columns=fullflexible,
keepspaces=true
}

\addbibresource{\jobname.bib}

\setlength\textheight{1.08\textheight}

\shorthandon{"}
\title{Notizen zum Projekt \\ "`VXI-11"=RPC als node.js"=Modul"'}
\date{2012-11-29}
\author{Rolf Niepraschk, PTB, AG~7.54}
\shorthandoff{"}

\begin{document}

\maketitle
\begin{abstract}
  \noindent
  Ziel des Projektes ist, eine Implementierung des VXI-11"=Protokolls
  (RPC"=basiert) in Form einer Javascript"=Bibliothek (Node.js"=Modul) zu
  erstellen und zu dokumentieren. Es soll eine möglichst einfache
  Möglichkeit geschaffen werden, basierend auf einer im Web"=Browser
  laufenden Javascript"=Applikation, einen Zugriff auf Messgeräte über
  das VXI-11"=Protokoll zu ermöglichen. Wenn möglich, sollte die fertige
  Software unter einer Freien Lizenz der Allgemeinheit zugänglich gemacht
  werden.
\end{abstract}

\section{RPC allgemein}

Folgende Quellen können genutzt werden, um Einzelheiten zu "`Remote
Procedure Call"' (RPC) zu erfahren:

\begin{itemize}
  \item \fullcite{bloomer:rpc}
  \item \fullcite{rpcgen:programming.guide}
  \item \fullcite{rpc:programming.guide}
  \item \fullcite{marshall:rpc}
  \item Quellen des Linux"=Kernels (\url{/usr/src/linux/net/sunrpc}) und
    zugehörige Dokumentationen.
\end{itemize}

\section{VXI-11}

Folgende Quellen können genutzt werden, um Einzelheiten zu "`VXI-11"'
zu erfahren:

\begin{itemize}
  \item \fullcite{vxibus:vxi-11}
  \item \fullcite{python:vxi-11}
  \item \fullcite{sharples:vxi-11}
\end{itemize}

\section{"`Node.js"' und Node.js"=RPC"=Module}

"`Node.js"' ist eine modulare serverseitige Javascript"=Umgebung, die die Basis
für die Umsetzung von http"=Zugriffen nach VXI-11 jetzt und auch künftig bildet. 

\begin{itemize}
  \item \fullcite{nodejs}
  \item \fullcite{nodejs:modules}
\end{itemize}

\end{document}
%---------------------------
