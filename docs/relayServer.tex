\listfiles
\documentclass[titlepage=false,toc=nobibliography]{vl-report}
% inputencoding ist fest auf utf8 eingestellt!

%%%\moveTypeArea% Heftrand erzwingen

\usepackage{hyperref}
\hypersetup{
  pdfborderstyle={/S/U/W 1}
}

\newcommand*\theServer{relayServer}
\newcommand*\action[1]{\fbox{\nolinkurl{#1}}\medskip\par}

\begin{document}

\subject{PTB~--~AG~7.54}
\title{Anleitung zu "`\theServer"'}
%\subtitle{xxx}
\date{2018-10-02}
\author{Rolf Niepraschk}

\maketitle

\section*{Einleitung}

Bei "`\theServer"' handelt es sich um ein auf "`nodejs"' basierendes
Programm, welches als sogenannter "`Dämon"' auf einem Linux"=Rechner
gestartet wird, d.\,h.\ es läuft solange auch der Linux"=Rechner läuft.
"`\theServer"' bietet Netzwerkzugänge in Form mehrerer
http"=Server auf unterschiedlichen Ports. Mit dieser Hilfe empfängt
"`\theServer"' Daten von außerhalb befindlichen Programmen (z.\,B.\
Web"=Applikationen) und sendet daraufhin verschiedenartige Daten zurück. Der
Sinn ist, auf diese Weise Funktionen anzubieten, die ohne "`\theServer"'
nicht oder nur in komplizierter Weise zur Verfügung stehen würden.

\section{relay"=Server -- Port: 55555}

Die Komponente "`relay"=Server"' bietet den Zugang zu einer Vielzahl von
externen auf dem Linux"=Rechner installierten Programmen und
Netzwerkprotokollen. Web"=Applikationen sind beispielsweise damit in der Lage
die sich aus der sogenannten "`Same-Origin-Policy"' ergebenen
Einschränkungen zu überwinden oder Daten per Netzwerkzugriff zu erhalten,
die normalerweise über diesen Weg nicht zugänglich sind.

Die Anforderung an den "`relay"=Server"' geschieht immer über eine
Datenstruktur im JSON"=Format. Diese enthält mindestens einen mit "`Action"'
benannten Wert, der die auszuführende Aktion auswählt.

\subsection{Allgemeingültige Parameter}

\begin{description}

  \item[Repeat] -- Positive Ganzzahl; gibt an, wie oft die betreffende
    Aktion wiederholt werden soll (optional, Standard: "`1"'). Dieser Parameter
    wird nicht von allen Aktionen unterstützt.

  \item[Wait] --  Positive Ganzzahl; gibt die Zeit in ms an, die nach Ende des
  vorherigen Ablaufs gewartet werden soll (optional, Standard: "`0"'). Diese
  Angabe ist nur wirksam, wenn "`Repeat"' größer als "`1"' ist. Manche Aktionen
  verwenden trotz "`Wait:0"' eine kleine Mindestwartezeit.
  
  \item[PostProcessing] -- String oder String-Array; JavaScript"=Code, mit dem 
  das Ergebnis einer vorher bereits beendeten Aktion nachbearbeitet werden 
  kann. Die Umgebung, in der diese Nachbearbeitung stattfindet, ist limitiert, 
  d.\,h.\ dass aus Sicherheitsgründen nur auf ausgewählte Komponenten 
  zugegriffen werden kann. Das sind im Einzelnen die folgenden Variablen:
  
  \begin{description}

    \item |_x|~\,~-- das noch unbearbeitete Ergebnis der vorherigen Aktion. 
    \item |_|~~~-- Objekt, welches den Zugriff auf die exportierten 
    Funktionen der Datei "`|relay-add.js|"' gestattet.
    \item |_$|~~-- Datenstruktur, die zum Anfang als JSON"=String an 
    "`\theServer"' geschickt wurde (enthält z.\,B.\ |Value| und |Action|),.
    
  \end{description}
  
  \noindent An die Rückgabe"=JSON"=Struktur werden alle die im JavaScript"=Code 
  definierten Variablen weitergereicht, die \emph{nicht} mit |_| (Unterstrich) 
  beginnen. Um den üblicherweise verwendeten Rückgabewert |Result| tatsächlich 
  zu ändern, ist z.\,B.\ eine solche Zuweisung nötig: 
  "`|Result= 3.14 * _x + 17;|"'
  
  \item[PreProcessing] -- String oder String-Array; JavaScript"=Code, mit dem 
  das aus dem Eingangs"=JSON"=String hervorgegangene Objekt gewandelt  
  werden kann. Dies geschieht sehr ähnlich zu dem Ablauf bei 
  "`|PostProcessing|"'. Im Gegensatz dazu ist die Variable |_$| nicht definiert 
  und die Variable |_x| enthält das erwähnte Eingangsobjekt. Eine Modifizierung 
  des Wertes von |Value| könnte z.\,B.\ folgendermaßen erfolgen: 
  "`|Value=_.encodeVACOM(_x.Value);|"'. ("`|encodeVACOM|"' ist eine per 
  "`|relay-add.js|"' bereitgestellte Funktion.) 

\end{description}

\subsection{Externe Aktionen}

Aufruf eines auf dem Server installierten Programms. Aus Gründen der Sicherheit 
geschehen diese Aufrufe -- wie auch der gesamte \theServer"=Prozess~-- mit 
Rechten des Users "`nobody"', d.\,h.\ mit sehr wenigen Rechten. 

\begin{description}
  \item \action{EXECUTE}
    \noindent Ausführen eines Programmes ggf. mit Paramtern, darunter auch zu 
    interpretierender Programmcode. Die Parameter:
    \begin{description}
      \item[Cmd] -- Name des auszuführenden Programmes mit komplettem Pfad 
      (zwingend).
      \item[Args0] -- Parameter (String oder Array von Strings). Wird für 
      den Programmaufruf direkt dem Programmnamen hinzugefügt (optional). 
      \item[Body] -- Text (meist Programmcode) in Form eines einfachen Strings 
      oder eines Arrays von Strings. Im zweiten Falle wird aus dem Array intern 
      ein einzelner String erzeugt, wobei nach jedem Array"=Element ein 
      Zeilenumbruch ("`|\n|"') eingefügt wird. Der gesamte Text wird beim 
      Ausführen dieser Aktion in eine Datei in einem temporären Verzeichnis 
      geschrieben. Der Name dieser Datei wird als Parameter dem Programmaufruf 
      hinzugefügt (optional). 
      \item[Args] -- Parameter (String oder Array von Strings). Folgen für den 
      Programmaufruf direkt dem Namen der temporären Datei. Bei per »Body« 
      angegebenem Programmcode können sie als Parameter für diesen genutzt 
      werden. Ohne »Body« verhält sich »Args« identisch zu »Args0« (optional).
      \item[KeepFiles] -- "`true"' oder "`1"' verhindert, dass nach erfolgtem 
      Programmaufruf das temporär angelegte Verzeichnis und dessen Inhalt 
      gelöscht wird (optional, Standard: "`false"', nur für Testzwecke). 
    \end{description}
\end{description}
Beispiel:
\begin{lstlisting}[language={}]
cat <<EOF | curl -T - -X PUT http://localhost:55555
{"Action":"EXECUTE","Cmd":"/usr/bin/date","Args":"+%Y-%m-%d"}
EOF
\end{lstlisting}
Die Antwort lautet:
\begin{lstlisting}[language={}]
{"t_start":1492593299399,"t_stop":1492593299409,
 "exitCode":0,"Result":"2017-04-19\n"}
\end{lstlisting}
Beispiel:
\begin{lstlisting}[language={}]
cat <<EOF | curl -T - -X PUT http://localhost:55555
{"Action":"EXECUTE","Cmd":"/usr/bin/Rscript",
"Body":["a <- 1:10","b <- which(a > 2 & a < 8)","b"],
"Args0":"--vanilla","Args":["foo","bar"]}
EOF
\end{lstlisting}
Das Senden dieses JSON"=Codes führt auf dem Server zum Anlegen einer Datei 
\verb|prg.dat| in einem temporären Verzeichnis mit folgendem Inhalt:
\begin{lstlisting}[language={},name=Beispiel:]
a <- 1:10
b <- which(a > 2 & a < 8)
b
\end{lstlisting}
Der komplette Programmaufruf im temporären Verzeichnis lautet:
\begin{lstlisting}[language={}]
/usr/bin/Rscript --vanilla prg.dat foo bar
\end{lstlisting} 
Die Parameter »foo« und »bar« deuten Programmparameter des R"=Programms an, die 
hier allerdings ignoriert werden. Zurückgesendet wird der vom aufgerufenen 
Programm auf die Standardausgabe ausgegebene Text ("`Result"'), in diesem 
Falle:
\begin{lstlisting}[language={},name=Rückgabe:]
{"t_start":1492589451825,"t_stop":1492589452055,
 "exitCode":0,"Result":"[1] 3 4 5 6 7\n"}
\end{lstlisting}

\subsection{Interne Aktionen}

  Es handelt sich um eine vom Server direkt ausführbare Aktion, bei der die
  Angabe des Namens eines externen Programms nicht notwendig ist bzw.\ es
  wird auf ein solches gänzlich verzichtet. Beginnt "`|Action|"' mit dem
  Zeichen "`|_|"', so kennzeichnet sie einen internen Prozess zu
  administrativen Zwecken.

 \begin{description}

  \item[Action] -- kennzeichnet die zu erledigende Aufgabe.

  \begin{description}

    \item \action{VXI11}
      \noindent
      Kommunikation über das VXI-11"=Protokoll mit Messgeräten. Die weiteren
      Parameter:

      \begin{description}

        \item[Host] -- IP-Adresse oder Rechnername des Messgerätes (zwingend)

        \item[Device] -- die Geräteadresse (zwingend)

        \item[Value] -- der String zum Auslösen eines bestimmten Gerätebefehls (zwingend)

        \item[Encoding] -- Format der Rückgabe (optional, Standard: utf8). "`base64"' sollte
          gewählt werden, wenn das Gerät Binärdaten schickt.

        \item[VxiTimeout] -- Alias für "`readTimeout"'
        \item[readTimeout] -- Zeit in ms, die bei einer Leseoperation auf eine
        Rückmeldung vom Gerät gewartet wird (optional, Standard: 2000). Der
        Wert "`0"' hat eine besondere Bedeutung:
        In diesem Falle wird beim Empfang der von dem angesprochenen Gerät
        gesendeten Daten ein "`timeout error"' ignoriert. Dieses Verhalten ist
        hilfreich bei Geräten, die tatsächlich Daten senden, aber fälschlich
        "`timeout"' signalisieren und bei Geräten, die überhaupt nichts
        zurücksenden.

        \item[ioTimeout] -- Zeit in ms für interne Operationen (optional, Standard: 10000)
        \item[lockTimeout] -- Zeit in ms für interne Operationen (optional, Standard: 10000)
        \item[lockDevice] -- Boolscher Wert, der bestimmt, ob der Zugriff auf das Gerät
          exklusiv ist oder nicht (optional, Standard: true).
        \item[termChar] -- Zahl oder String (optional, Standard: 0 -- no term char)

      \end{description}

\begin{lstlisting}[language={},name=Beispiel:]
echo '{"Action":"VXI11","Host":"e75481",
  "Device":"gpib0,5","Value":"*IDN?"}' | \
  curl -T - -X PUT http://localhost:55555
\end{lstlisting}

\begin{lstlisting}[language={},name=Rückgabe:]
{"t_start":1401797586696,"t_stop":1401797586737,
  "Result":"HEWLETT-PACKARD,34970A,0,13-2-2\n"}
\end{lstlisting}

      Die vom Gerät gesendete Antwort ist dem Kennwort "`Result"'
      zugeordnet. "`|t_start|"'/"""`|t_stop|"' kennzeichnen die
      Dauer des Aufrufes in ms. Falls die absolute Zeit
      von Interesse ist, kann sie mit
%
\begin{lstlisting}[language={}]
date -d @$[1401797586696 / 1000]
\end{lstlisting}
%
      in Sekundengenauigkeit dargestellt werden: \par
      \quad |Di 3. Jun 14:13:06 CEST 2014|.

    \item \action{MODBUS}

      Der "`\theServer"' verhält sich in diesem Modus wie ein
      Modbus-TCP"=Server.

      \begin{description}

        \item[Host] -- IP-Adresse oder Rechnername des Modbus"=Clients
        (zwingend)

        \item[Port] -- Port-Nummer (optional, Standard: 502)

        \item[FunctionCode] -- Modbus"=Funktion. Derzeit (2015-10-22) werden
        nur die folgenden unterstützt:

          \begin{itemize}

            \item ReadHoldingRegisters
            \item WriteSingleRegister
            \item ReadInputRegisters
            \item ReadCoils
            \item ReadDiscreteInputs
            \item WriteSingleCoil

          \end{itemize}
        Ausführlich getestet wurden nur die beiden ersten.

        \item[Address] -- Anfangsadresse des zu erfragenden Speichers im
        Modbus"=Client (zwingend)

        \item[Quantity] -- Anzahl der auszulesenden aufeinanderfolgenden
        Speicherzellen (optional, Standard: 1, Maximum: 125). Ist
        $\text{Quantity}>1$, wird ein Array als Antwort ("`Result"')
        zurückgeliefert. Im anderen Fall ist es ein Einzelwert.

        \item[Value] -- Der Wert, der in eine Speicherzelle geschrieben werden
        soll (bei Schreiboperationen zwingend). Bei numerischen Operationen
        (z.\,B.\ "`WriteSingleRegister"') wird eine Ganzzahl verlangt. Bei
        diskreten Operationen (z.\,B.\ "`WriteSingleCoil"') wird ein boolscher
        Wert ("`true"' oder "`false"') erwartet. Bei Angabe einer Zahl wird
        diese als "`true"' interpretiert, es sei denn, es handelt sich um den
        Wert "`0"'. Ist "`Value"' ein Array, so wird angenommen, dass es sich
        um ein Bit-Array mit 0- oder 1-Werten handelt (Index 15 $=$ Wertigkeit
        16384).

        \item[OutMode] -- Gibt an, in welchem Format das Ergebnis geliefert
        werden soll (optional, Standard: "`Uint16"'). Ist nur bei den
        Modbus"=Funktion, die numerische Werte liefern, wirksam
        ("`ReadHoldingRegisters"' und "`ReadInputRegisters"'). Folgende Angaben
        sind möglich:

          \begin{description}

            \item[Uint16] -- Array of 16Bit-Integers oder einzelner 16Bit-Wert
            \item[16Bits] -- Array of Bit-Arrays (16 Bits) \par\noindent
              Mit "`Bit-Array"' ist ein numerisches Feld mit 0- oder 1-Werten
              (Ja/Nein-Werten) gemeint.

            \item[8Bits] -- Array of Bit-Arrays (8 Bits) \par\noindent
              Es werden nur die niederwertigen 8 Bit der jeweiligen
              16-Bit"=Zahl als Bit-Array dargestellt.

            \item[16Bits*] -- Bit-Array (Quantity * 16 Bits)\par\noindent
              Es werden sämtliche Bits aller 16-Bit"=Zahlen als ein Bit-Array
              dargestellt.

            \item[8Bits*] -- Bit-Array (Quantity * 8 Bits) \par\noindent
              Es werden nur die niederwertigen 8 Bit aller
              16-Bit"=Zahlen als ein Bit-Array dargestellt.

          \end{description}

        \item[Skip] -- Gibt an, wieviele der erfragten 16-Bit"=Werte
        übersprungen werden, bevor ein weiterer als Ausgabewert verwendet wird
        (optional, Standard: 0; derzeit ist nur der Wert "`1"' wirksam). Der
        Parameter ist dann nützlich, wenn sich in regulär fortlaufender Folge
        wichtige und unwichtige Werte abwechseln. Jene können auf diese Art
        unterdrückt werden.

      \end{description}

\noindent Die folgenden Beispiele zeigen die Abfrage der digitalen Eingänge von
sechs Modulen "`8DE"' der FESTO"=Ventilinsel (Modul~1: 24V an Eingang~1,
Modul~2: 24V an Eingang~2, \dots, Modul~6: 24V an Eingang~6). Jeder zweite
16-Bit"=Wert enthält sogenannte "`Diagnosedaten"', die hier nicht von
Interesse sind und daher im zweiten und dritten Beispiel ausgeblendet werden.

\begin{lstlisting}[language=bash]
cat <<EOF | curl -T - -X PUT http://localhost:55555
{"Action":"MODBUS","Host":"10.0.0.31","Address":45395,"Quantity":11,
 "FunctionCode":"ReadHoldingRegisters", "OutMode":"Uint16"}
EOF
# ======== Ergebnis: ========
{
  "t_start": 1445500306261,
  "t_stop": 1445500306261,
  "Result": [1,0,2,0,4,0,8,0,16,0,32]
}
\end{lstlisting}

\begin{lstlisting}[language=bash]
cat <<EOF | curl -T - -X PUT http://localhost:55555
{"Action":"MODBUS","Host":"10.0.0.31","Address":45395,"Quantity":11,
 "FunctionCode":"ReadHoldingRegisters", "OutMode":"8Bits","Skip":1}
EOF
# ======== Ergebnis: ========
{
  "t_start": 1445500592190,
  "t_stop": 1445500592197,
  "Result": [
    [1,0,0,0,0,0,0,0],
    [0,1,0,0,0,0,0,0],
    [0,0,1,0,0,0,0,0],
    [0,0,0,1,0,0,0,0],
    [0,0,0,0,1,0,0,0],
    [0,0,0,0,0,1,0,0]
  ]
}
\end{lstlisting}

\begin{lstlisting}[language=bash]
cat <<EOF | curl -T - -X PUT http://localhost:55555
{"Action":"MODBUS","Host":"10.0.0.31","Address":45395,"Quantity":11,
 "FunctionCode":"ReadHoldingRegisters", "OutMode":"8Bits*","Skip":1}
EOF
# ======== Ergebnis: ========
{
  "t_start": 1445500905697,
  "t_stop": 1445500905724,
  "Result":
  [1,0,0,0,0,0,0,0,0,1,0,0,0,0,0,0,0,0,1,0,0,0,0,0,0,0,0,1,0,0,0,0,
   0,0,0,0,1,0,0,0,0,0,0,0,0,1,0,0]
}
\end{lstlisting} 

\noindent Es folgen Beispiele, die Schreiboperationen zeigen. Mit 40007 wird 
das Register einer FESTO-Ventilinsel adressiert, welches bei uns zum Schalten 
einzelner Ventile benutzt wird. Soll z.\,B.\ das 5te Ventil eingeschaltet werden, 
muss das 5te Bit den Wert "`1"' erhalten. Im folgenden Beispiel werden alle 16 
Bits auf "`1"' gesetzt ($2^{16}-1 = 65535$):

\begin{lstlisting}[language=bash]
cat <<EOF | curl -T - -X PUT http://localhost:55555
{"Action":"MODBUS","Host":"172.30.56.46","Address":40007,
"FunctionCode":"WriteSingleRegister", "Value":65535}
EOF
\end{lstlisting}

\noindent Alternativ kann auch ein Array, welches 16 "`0"'- oder "`1"'-Werte 
enthält, gesendet werden:

\begin{lstlisting}[language=bash]
cat <<EOF | curl -T - -X PUT http://localhost:55555
{"Action":"MODBUS","Host":"10.0.0.31","Address":40007,
"FunctionCode":"WriteSingleRegister",
"Value":[0,0,0,0,1,0,0,0,0,0,0,0,0,0,0,0]}
EOF 
\end{lstlisting}


    \item \action{TEX}

      Bei dieser Aktion handelt es sich eigentlich auch um einen externen
      Programmauf. Aufgrund der Komplexität ist sie dennoch als "`interne
      Aktionen"' eingeordnet.

      Es wird aus dem übergebenen \TeX"=Code eine Datei erzeugt, die mit
      einem der unterstützten \TeX"=Compiler in PDF"=Code (PDF"=Stream)
      konvertiert wird.

      \begin{description}

        \item[Body] -- \TeX"=Code in Form eines Strings oder eines Arrays von
        Strings (zwingend). Im zweiten Falle wird aus dem Array intern ein
        einzelner String erzeugt, wobei nach jedem Inhalt eines
        Array"=Elements ein Zeilenumbruch ("`|\n|"') eingefügt wird.

        \item[Command] -- Der Name des \TeX"=Compilers (optional, Standard:
        "`pdflatex"'). Es werden nur \TeX"=Compiler unterstützt, die auf
        direktem Wege PDF"=Dateien erzeugen, d.\,h.\ "`pdflatex"',
        "`lualatex"', "`xelatex"', "`pdftex"', "`luatex"', "`xetex"'.

        \item[KeepFiles] -- "`true"' oder "`1"' verhindert, dass nach
        erfolgtem Programmaufruf das temporär angelegte Verzeichnis und
        dessen Inhalt gelöschte wird (optional, Standard: "`false"', nur für
        Testzwecke).

      \end{description}

    \item \action{TCP}

      \begin{description}

        \item[Host] -- IP-Adresse oder Rechnername (zwingend)

        \item[Port] -- Port-Nummer (zwingend)

        \item[Value] -- Zu sendende Daten (zwingend)

        \item[Timeout] -- Zeit in ms, die bei einer Leseoperation auf eine
        Rückmeldung vom Gerät gewartet wird (optional, Standard: 10000).

      \end{description}

      \noindent Die Antwort der Gegenstelle (|"Result"|) ist ein String.


    \item \action{HTTP}

      \begin{description}

        \item[URL] -- (zwingend)

        \item[Body] -- Zu sendende Daten (optional)

        \item[Method] -- HTTP"=Methode ("`GET"', "`POST"' oder "`PUT"';
        optional). Ist sie nicht angegeben, so wird im Falle von vorhandenem
        Body"=Eintrag "`POST"' gewählt, sonst "`GET"'.

        \item[Json] -- "`true"' oder "`false"' (optional, Standard: false).
        \item[Headers] -- "`|{}|"' (optional)
        \item[Encoding] -- (optional, Standard: utf8).
        \item[Timeout] -- (optional, Standard: 20...120\,sec).

      \end{description}
Beispiel:
\begin{lstlisting}[language={}]
cat <<EOF | curl -T - -X PUT http://localhost:55555
{"Action":"HTTP","Url":"http://a73434:55555",
"Body":{"Action":"_os_release"},"Json":true}
EOF
\end{lstlisting}
Es wird dem lokalen "`\theServer"' der Auftrag erteilt, per HTTP"=Protokoll 
einen anderen entfernten "`\theServer"' anzusprechen und mit diesem die Action  
"`|_os_release|"' auszuführen. Die zurückgelieferten JSON"=Daten
\begin{lstlisting}[language={}]
{"t_start":1538464194174,"t_stop":1538464194184,
"Result":{"Result":{"NAME":"openSUSE Leap",
"VERSION":"42.3","ID":"opensuse","ID_LIKE":"suse",
"VERSION_ID":"42.3","PRETTY_NAME":"openSUSE Leap 42.3",
"ANSI_COLOR":"0;32","CPE_NAME":"cpe:/o:opensuse:leap:42.3",
"BUG_REPORT_URL":"https://bugs.opensuse.org",
"HOME_URL":"https://www.opensuse.org/"}}}
\end{lstlisting}
enthalten in diesem Falle Angaben zum Betriebssystem des entfernten Rechners.

    \item \action{UDP}

      \begin{description}

        \item[Host] -- IP-Adresse oder Rechnername (zwingend)

        \item[Port] -- Port-Nummer (zwingend)

        \item[Value] -- Zu sendende Daten (zwingend)

      \end{description}

      \noindent Die Antwort (|"Result"|) ist, abgesehen von drastischen
      Fehlerfällen, immer der String |"OK"|.

    \item \action{EMAIL}

      \begin{description}

        \item[Host] -- SMTP-Server (zwingend, z.\,B.\ |smtp-hub|)

        \item[Port] -- Port-Nummer (optional, Standard: 25)

        \item[From] -- Muss wie Email-Adresse aussehen (zwingend)
        
        \item[ReplyTo] -- Gültige Antwort-Email-Adresse (optional)

        \item[To] -- Gültige Email-Adresse (zwingend)

        \item[Text] -- Zu sendender einfacher Text (optional, 
          String oder String-Array, Standard: |""|)
        
        \item[Html] -- Zu sendender HTML-Code (optional und alternativ, 
          String oder String-Array, Standard: |""|)
        
        \item[Subject] -- Betreff (optional, aber dringend empfehlenswert)

        \item[Cc, Bcc] -- Verteiler (optional)
        
        \item[Secure] -- \verb+true|false+ (optional, Standard: |false|)

      \end{description}

      \noindent Zu diesen und weiteren Parametern, die teilweise noch ungetestet 
      sind, siehe: \url{http://www.nodemailer.com/}. Beispiel:
\begin{lstlisting}[language={}]
cat <<EOF | curl -T - -X PUT http://localhost:55555
{"Action":"EMAIL", "Host":"smtp-hub", "From": "VacLab <invalid@ptb.de>", 
 "To":"Rolf.Niepraschk@ptb.de", "ReplyTo":"Thomas.Bock@ptb.de",
 "Subject":"Von Diwan", "Text":"Hallo Rolf!"}
EOF
\end{lstlisting}

    \item \action{XLSX-OUT}

      \noindent Eine JSON-Datenstruktur, die ein oder mehrere Arbeitsblätter
      mit Datenfeldern enthält, wird in das XML"=basierte EXCEL"=Datenformat
      gewandelt und als Binär"=Stream zurückgesendet.

      \begin{description}

        \item[Value] -- JSON"=Objekt mit enthaltenen Arbeitsblättern. Das
        JSON"=Objekt muss folgende Struktur haben:
\begin{lstlisting}[language={}]
cat <<EOF | curl -T - -X PUT http://localhost:55555 > hugo.xlsx
{"Action":"XLSX-OUT",
  "Value":[
    {
      "name": "worksheet 1",
      "data": [
        ["A1", "B1", "C1"],
        ["A2", "B2", "C2"]
      ]
    }, 
    {
      "name": "worksheet 2",
      "data": [
        ["X1", "Y1", "Z1"],
        ["X2", "Y2", "Z2"]
      ]
    }
  ]
}
EOF
\end{lstlisting}
      Als Datentypen werden Strings und numerische Werte unterstützt.
      \end{description}

    \item \action{XLSX-IN}

      \noindent Der BASE64"=kodierte Inhalt einer XML"=basierten EXCEL"=Datei
      wird in eine JSON-Datenstruktur (siehe "`XLSX-OUT"') gewandelt und
      selbige zurückgesendet.

      \begin{description}

        \item[Value] -- EXCEL"=Daten als BASE64"=kodierter String

      \end{description}
\begin{lstlisting}[language=bash]
cat <<EOF | curl -T - -X PUT http://localhost:55555
{"Action":"XLSX-IN","Value":"$(base64 -w 0 foo.xlsx)"}
EOF
# ======== Ergebnis: ========
{
  "t_start": 1475227458498,
  "t_stop": 1475227458548,
  "Result": [{
    "name": "Tabelle1",
    "data": [
      ["Spalte X", "Spalte Y", "Spalte Z"],
      ["X1", "Y1", "Z1"],
      ["X2", "Y2", "Z2"],
      ["X3", "Y3", "Z3"],
      [117, 118, 119]
    ]
  }]
}
\end{lstlisting}

    \item \action{RANDOM}

      \noindent Es gibt hier keine weiteren Parameter. Die Antwort
      (|"Result"|) ist eine Fließkommazahl (Stringform) zwischen 0 und 1.

    \item \action{TIME}

      \noindent Es gibt hier keine weiteren Parameter. Die Antwort
      (|"Result"|) ist eine Zeitangabe in der Form |"HH:MM:SS"|.

    \item \action{_killRepeats}

      \noindent Es gibt hier keine weiteren Parameter. Ziel dieser Aktion
      ist es, alle anderen vorher laufenden Aktionen, die sich jeweils in einer
      durch |"Repeat"| festgelegten Schleife befinden, vorfristig zu beenden.

    \item \action{_version}

      \noindent Es gibt hier keine weiteren Parameter. Die Antwort
      (|"Result"|) ist die Angabe zur Version des laufenden \Program
      {Node.js}"=Relais"=Servers sowie der \Program{nodejs}-Version.

    \item \action{_nodejsVersion}

       \noindent Es gibt hier keine weiteren Parameter. Die Antwort
       (|"Result"|) ist die Angabe zur Version des \Program
       {Node.js}"=Programms.

    \item \action{_environment}

      \noindent Es gibt hier keine weiteren Parameter. Die Antwort
      (|"Result"|) ist die Angabe der dem \Program{Node.js}"=Relais"=Server
      bekannten Umgebungsvariablen des Betriebssystems in Form eines
      JSON-Objektes.

    \item \action{_exit}

      \noindent Es gibt hier keine weiteren Parameter. Die Antwort (|"Result"|)
      ist |"OK"| und der Rückkehrcode 0. Der Server wird direkt nach Senden der
      Antwort beendet. Sofern dafür Sorge getragen wird, dass der
      "`\theServer"' sofort wieder neu gestartet wird (z.\,B.\ per
      "`systemd"'), kann »|_exit|« dazu verwendet werden, eine erneuerte
      Programmversion wirksam werden zu lassen.

  \end{description}

\end{description}

\section{Gitlab"=Hook"=Server -- Port: 3420}

\minisec{Anmerkung:} Die im Folgenden beschriebene Serverfunktionalität ist
nicht mehr innerhalb des zuvor beschriebenen  "`\theServer"' enthalten
(2014-10-08). Zu einem späteren Zeitpunkt wird es ein eigenständiges Paket mit
dem Gitlab"=Hook"=Server geben.

\bigskip

\noindent Die Web"=Applikation "`GitLab"' gestattet ähnlich zu dem Konkurrenzprodukt
"`GitHub"' einen bequemen Zugriff auf das Versionskontrollsystem "`GIT"'. Zu
jedem  in "`GitLab"' registrierten Repositorium kann über
"`Settings"'/"""`Web hooks"' eine http"=Adresse angegeben werden. Nach
Änderung des betreffende Repositoriums (push"=Aktion) wird an diese Adresse
eine Datenstruktur gesendet. Sie enthält etliche Angaben zu dem
Repositorium, wie Name des Repositoriums, URL für GIT"=Aktionen, Nutzername,
IDs der letzten Commits u.\,v.\,a. "`\theServer"' bietet die
Funktionalität eines Gitlab"=Hook"=Servers und ist somit in der Lage, diese
Informationen von "`GitLab"' zu empfangen und auszuwerten. Die Adresse, die
in "`GitLab"' angegeben werden muss, lautet im Falle des auf "`a73434"'
laufenden "`\theServer"' folgendermaßen:
\begin{lstlisting}[language={}]
  http://a73434.berlin.ptb.de:3420
\end{lstlisting}
Zur Definition dessen, was nach Eintreffen der
Informationen von "`GitLab"' zu tun ist, muss eine Datei |gitlabhook.conf|
angelegt werden. Als Beispiel sei hier der Inhalt angeführt, der dazu führt,
dass sich automatisch mit Änderung des Repositoriums "`vaclabpage"'
html"=Seiten, die vom Webserver ausgeliefert werden, erneuern:
\begin{lstlisting}[language={}]
{
  "tasks": {
    "vaclabpage": [
      "exec 1>/dev/null",
      "exec 2>/dev/null",
      "git clone %h",
      "cd %r",
      "cp -p --parents $(git ls-files) /srv/www/htdocs/vaclabpage/"
    ]
  },
  "keep":false
}
\end{lstlisting}%//$
%
Zur Erklärung: Unter "`tasks"' können ein oder mehrere Namen von
GIT"=Repositorien aufgeführt werden. Jedem dieser Namen ist ein String oder
ein String"=Array zugeordnet. Darin enthalten sind Unix"=Kommandozeilenaufrufe.
"`|%h|"' ist ein Platzhalter für die zum Clonen des GIT"=Repositorium
verwendbare URL. "`|%r|"' steht für den Namen des Repositoriums. Mit
"`|"keep":false|"' wird festgelegt, dass temporär erstellte Verzeichnisse
nicht erhalten bleiben sollen (|"true"| ist nur für Testzwecke nützlich). Die
Beschreibung zu dem NodeJS"=Module "`node-gitlab-hook"' enthält weitere
Hinweise zur Syntax in |gitlabhook.conf|: \par\smallskip
\url{https://github.com/rolfn/node-gitlab-hook} \par\smallskip
\noindent
Der konkrete Ablauf im temporären Verzeichnis des Rechners
(|/tmp|) zur Erneuerung der Home"=Page des Vakuumlabors ist also der
folgende:
\begin{enumerate}
  \item Lokales Duplikat des GIT"=Repositoriums anlegen.
  \item In das Verzeichnis mit dem Namen des Repositoriums wechseln.
  \item Alle relevanten Dateien zum Webserver"=Verzeichnis kopieren.
\end{enumerate}
Es ist zu beachten, dass alle unter "`\theServer"' laufenden Prozesse
im Falle von openSUSE mit den Rechten des Nutzers "`wwwrun"' laufen.

\section{Logging"=Zugriff per Websocket"=Protokoll -- Port: 9001}

Um detaillierte Informationen über den internen Ablauf beim Ansprechen der
unter "`\theServer"' laufenden Server"=Prozesse zu erhalten, kann über
Port~|9001| per Websocket"=Protokoll Kontakt aufgenommen werden. Zu
diesem Zweck steht das Kommandozeilen"=Programm |vlLogging| zur Verfügung.
Ohne Parameter nimmt es Kontakt zum lokal laufenden "`\theServer"' auf.
Wird als Parameter ein entfernter Rechner (Rechnername oder IP"=Adresse)
angegeben, kann auch auf dessen Informationen zugegriffen
werden. Die Anzahl der kontaktierten Prozesse ist nicht beschränkt. Es wird
das folgende Ausgabeformat geliefert:
\begin{lstlisting}[language={}]
2013-10-17 08:53:10.862 - LEVEL: [FILE_NAME:LINE_NUMBER:FUNCTION_NAME] MESSAGE
\end{lstlisting}
\begin{itemize}
  \item |LEVEL|: "`debug"', "`error"', "`warn"' oder "`info"'
  \item |FILE_NAME|: Die Datei, in der sich der abgearbeitete Code befindet.
  \item |LINE_NUMBER|: Aktuelle Zeilennummer in |FILE_NAME|
  \item |FUNCTION_NAME|: Funktion, in der sich der abgearbeitete Code befindet.
  \item |MESSAGE|: Konkrete Informationen zum betreffenden Code"=Teil.
\end{itemize}
%
Diese Informationen dienen der Fehlersuche im Programmcode von
"`\theServer"' und auch zur allgemeinen Beobachtung des Ablaufes
beispielsweise bei der Kommunikation mit Messgeräten. Die Angabe |LEVEL|
wird farblich hervorgehoben. Die rot gekennzeichneten Fehlermeldungen fallen
besonders auf. Künftig könnte der Websocket"=Zugriff auch von einer
Web"=Applikation (Web"=Browser) aus erfolgen.

\mbox{}\vfill\par\bigskip
\begingroup \small \itshape

\noindent Hinweise zu Verbesserungen dieses Dokuments bitte als
Issue"=Eintrag des git"=Projektes "`\theServer"' (siehe "`GitLab"') oder per
E-Mail an \url{Rolf.Niepraschk@ptb.de}.

\endgroup

\end{document}
%---------------------------


%---------------------------
\setcounter{errorcontextlines}{100}
\listfiles
\documentclass[%
fontsize=11pt
,paper=a4
,twoside
,headings=normal
,numbers=endperiod
,pagesize
]{scrartcl}
\usepackage{hyphsubst}% Trennregeln austauschen
\HyphSubstIfExists{ngerman-x-latest}{%
  \HyphSubstLet{ngerman}{ngerman-x-latest}}{}

\usepackage[utf8]{inputenc}

\usepackage{cmap}
\usepackage[T1]{fontenc}
\usepackage{lmodern}
\usepackage[largesmallcaps]{kpfonts}
\usepackage[scale=.85]{tgheros}
%%\renewcommand{\sfdefault}{cmbr}
\usepackage[scaled]{luximono}
\usepackage{microtype}
\DeclareMicrotypeAlias{jkp}{ppl}
\DeclareMicrotypeAlias{jkpk}{ppl}
\usepackage[obeyspaces,spaces]{url}
\usepackage[ngerman]{babel}
\usepackage[autostyle=true,babel=once,german=guillemets,maxlevel=3]{csquotes}%
\defineshorthand{"`}{\openautoquote}
\defineshorthand{"'}{\closeautoquote}

%\setkomafont{minisec}{\normalfont\bfseries}
%\setkomafont{disposition}{\normalfont\bfseries}
%\setkomafont{descriptionlabel}{\normalfont\bfseries}
%\usepackage[scale=.85]{tgheros}
\usepackage{tgheros}
\DeclareMicrotypeAlias{jkp}{ppl}
\DeclareMicrotypeAlias{jkpk}{ppl}

\renewcommand*\labelitemii{$\circ$}

\makeatletter
\newcommand\Minisec[1]{\@afterindentfalse \vskip 1.5ex
  {\parindent \z@
    \raggedsection\normalfont\sectfont\nobreak
    \usekomafont{minisec}#1\nobreak}\nobreak%
  \@afterheading
}
\makeatother

\usepackage{xcolor,listings}
\lstset{%
     basicstyle=\footnotesize\ttfamily,
     identifierstyle={},
     keywordstyle=\bfseries,
     %stringstyle=\itshape\color{DTKlstStrings},
     commentstyle=\itshape,
     columns=fixed,
     tabsize=2,
     frame=single,
     extendedchars=true,
     showspaces=false,
     showstringspaces=false,
     breaklines=true,
     breakindent=10pt,
     backgroundcolor=\color{black!10},
     breakautoindent=true,
     captionpos=t,
     aboveskip=\medskipamount,
     belowskip=\medskipamount,
     xrightmargin=\fboxsep,
     prebreak=\mbox{$\hookleftarrow$},
     columns=fullflexible,
     keepspaces=true,
     title={\lstname},
}
\makeatletter
\def\lst@maketitle#1{%
   \vskip\abovecaptionskip
     #1\par
   \vskip\belowcaptionskip}%
\makeatother

\setlength\textheight{1.08\textheight}

\shorthandon{"}
\title{Gebrauchsanweisung zu "`\theServer"'}
\date{2014-01-27}
\author{Rolf Niepraschk, AG~7.54}
\shorthandoff{"}

\usepackage{shortvrb}
\MakeShortVerb{\|}
