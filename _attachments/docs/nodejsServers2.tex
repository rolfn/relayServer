%---------------------------
\setcounter{errorcontextlines}{100}
\listfiles
\documentclass[%
fontsize=11pt
,paper=a4
,twoside
,headings=normal
,numbers=endperiod
,pagesize
]{scrartcl}
\usepackage{hyphsubst}% Trennregeln austauschen
\HyphSubstIfExists{ngerman-x-latest}{%
  \HyphSubstLet{ngerman}{ngerman-x-latest}}{}

\usepackage[utf8]{inputenc}

\usepackage{cmap}
\usepackage[T1]{fontenc}
\usepackage{lmodern}
\usepackage[largesmallcaps]{kpfonts}
\usepackage[scale=.85]{tgheros}
%%\renewcommand{\sfdefault}{cmbr}
\usepackage[scaled]{luximono}
\usepackage{microtype}
\DeclareMicrotypeAlias{jkp}{ppl}
\DeclareMicrotypeAlias{jkpk}{ppl}
\usepackage[obeyspaces,spaces]{url}
\usepackage[ngerman]{babel}
\usepackage[autostyle=true,babel=once,german=guillemets,maxlevel=3]{csquotes}%
\defineshorthand{"`}{\openautoquote}
\defineshorthand{"'}{\closeautoquote}

%\setkomafont{minisec}{\normalfont\bfseries}
%\setkomafont{disposition}{\normalfont\bfseries}
%\setkomafont{descriptionlabel}{\normalfont\bfseries}
%\usepackage[scale=.85]{tgheros}
\usepackage{tgheros}
\DeclareMicrotypeAlias{jkp}{ppl}
\DeclareMicrotypeAlias{jkpk}{ppl}

\renewcommand*\labelitemii{$\circ$}

\makeatletter
\newcommand\Minisec[1]{\@afterindentfalse \vskip 1.5ex
  {\parindent \z@
    \raggedsection\normalfont\sectfont\nobreak
    \usekomafont{minisec}#1\nobreak}\nobreak%
  \@afterheading
}
\makeatother

\usepackage{xcolor,listings}
\lstset{%
     basicstyle=\footnotesize\ttfamily,
     identifierstyle={},
     keywordstyle=\bfseries,
     %stringstyle=\itshape\color{DTKlstStrings},
     commentstyle=\itshape,
     columns=fixed,
     tabsize=2,
     frame=single,
     extendedchars=true,
     showspaces=false,
     showstringspaces=false,
     breaklines=true,
     breakindent=10pt,
     backgroundcolor=\color{black!10},
     breakautoindent=true,
     captionpos=t,
     aboveskip=\medskipamount,
     belowskip=\medskipamount,
     xrightmargin=\fboxsep,
     prebreak=\mbox{$\hookleftarrow$},
     columns=fullflexible,
     keepspaces=true,
     title={\lstname},
}
\makeatletter
\def\lst@maketitle#1{%
   \vskip\abovecaptionskip
     #1\par
   \vskip\belowcaptionskip}%
\makeatother

\setlength\textheight{1.08\textheight}

\shorthandon{"}
\title{Gebrauchsanweisung zu "`nodejsServers2"'}
\date{2014-01-27}
\author{Rolf Niepraschk, AG~7.54}
\shorthandoff{"}

\usepackage{shortvrb}
\MakeShortVerb{\|}

\begin{document}

\maketitle

\section*{Einleitung}

Bei "`nodejsServers2"' handelt es sich um ein auf "`nodejs"' basierendes
Programm, welches als sogenannter "`Dämon"' auf einem Linux"=Rechner
gestartet wird, d.\,h.\ es läuft solange auch der Linux"=Rechner läuft.
"`nodejsServers2"' bietet einen Netzwerkzugänge in Form mehrerer
http"=Server auf unterschiedlichen Ports. Mit dieser Hilfe empfängt
"`nodejsServers2"' Daten von außerhalb befindlichen Programmen (z.\,B.\
Web"=Applikationen) und sendet daraufhin verschiedenartige Daten zurück. Der
Sinn ist, auf diese Weise Funktionen anzubieten, die ohne "`nodejsServers2"'
nicht oder nur in komplizierter Weise zur Verfügung stehen würden.

\section{relay"=Server -- Port: 55555}

Die Komponente "`relay"=Server"' bietet den Zugang zu einer Vielzahl von
externen auf dem Linux"=Rechner installierten Programmen und
Netzwerkprotokollen. Web"=Applikationen sind beispielsweise damit in der Lage
die sich aus der sogenannten "`Same-Origin-Policy"' ergebenen
Einschränkungen zu überwinden oder Daten per Netzwerkzugriff zu erhalten,
die normalerweise über diesen Weg nicht zugänglich sind.

Die Anforderung an den "`relay"=Server"' geschieht immer über eine
Datenstruktur im JSON"=Format. Diese enthält mindestens einen mit "`Action"'
benannten Wert, der die auszuführende Aktion auswählt. Sogenannte "`externe
Aktionen"' beginnen mit dem Zeichen "`/"' und bilden den Dateipfad zu einem
auf dem Rechner installierten Programm, welches gestartet werden soll.
Es sind nur die im Folgenden aufgeführten Programme erlaubt. Aktionen, die
nicht das Zeichen "`/"' enthalten, werden als "`interne Aktionen"'
bezeichnet, da sie mit einer Ausnahme ohne Start eines besonderen Programms
auskommen, also einzig die Möglichkeiten der Programmiersprache
"`JavaScript"' in der durch "`nodejs"' angebotenen Form nutzen.

\subsection{Externe Aktionen}

\begin{description}

  \item[Action] -- kennzeichnet das Programm, welches von "`NodejsRelay"'
  gestartet wird.

  \begin{itemize}

    \item |"/usr/local/bin/vxiTransceiver"|: \par

      Kommunikation über das VXI-11"=Protokoll mit Messgeräten. Die weiteren
      Parameter:

      \begin{description}

        \item[Host] -- IP-Adresse oder Rechnername des Messgerätes (zwingend)

        \item[Device] -- die Geräteadresse (zwingend)

        \item[Value] -- der String zum Auslösen eines bestimmten Gerätebefehls (zwingend)

        \item[VxiTimeout] -- Zeit in ms, die auf eine Rückmeldung vom Gerät
        gewartet wird (optional, Standard: 2000). Der Wert "`0"' kann
        benutzt werden, um bei fehlerhaften GPIB"=Geräten einen "`timeout
        error"' zu vermeiden, wenn diese zwar Daten aber kein korrektes
        Ende"=Signal senden.

      \end{description}

\begin{lstlisting}[language={},name=Beispiel:]
echo '{"Action":"/usr/local/bin/vxiTransceiver","Host":"e75481",
  "Device":"gpib0,5","Value":"*IDN?"}' | \
  curl -T - -X PUT http://localhost:55555
\end{lstlisting}

\begin{lstlisting}[language={},name=Rückgabe:]
{
"t_start":1385556963144,"t_stop":1385556963157,
"exitCode":0,
"t__start":1385556963131,"t__stop":1385556963161,  ???
"Result":["HEWLETT-PACKARD,34970A,0,13-2-2\n"]
}
\end{lstlisting}

      Die vom Gerät gesendete Antwort ist dem Kennwort "`Result"'
      zugeordnet. "`exitCode"' ist der Rückgabecode des Programmaufrufs (in
      der Shell "`|$?|"') und "`|t_start|"'/"""`|t_start|"' kennzeichnen die
      Dauer des Aufrufes in ms. Falls die absolute Zeit
      von Interesse ist, kann sie mit

\begin{lstlisting}[language={}]
date -d @$[1385556963157 / 1000]
\end{lstlisting}

      in Sekundengenauigkeit dargestellt werden: \par
      \quad |Mi 27. Nov 13:56:03 CET 2013|.

    \item |"/usr/bin/Rscript"|: \par

      Senden von Programmcode an die Statistik"=Software "`R"'. Die weiteren
      Parameter:
      \begin{description}

        \item[Body] -- R-Programmcode in Form eines Strings oder eines
        Arrays von Strings (zwingend). Im zweiten Falle wird aus dem Array
        intern ein einzelner String erzeugt, wobei nach jedem Inhalt eines
        Array"=Elements ein Zeilenumbruch ("`|\n|"') eingefügt wird.

        \item[Value] -- Zusätzliche Parameter für den Aufruf des Programms
        "`Rscript"' (optional, String oder String"=Array). Der Aufruf
        enthält später in jedem Falle als ersten Parameter vor den hier
        angegebenen den Namen der temporären Datei mit dem Programmcode.

        \item[KeepFiles] -- "`true"' oder "`1"' verhindert, dass nach
        erfolgtem Programmaufruf das temporär angelegte Verzeichnis und
        dessen Inhalt gelöschte wird (optional, Standard: "`false"', nur für
        Testzwecke).

      \end{description}
\enlargethispage{-2\baselineskip}
\begin{lstlisting}[language={},name=Beispiel:]
cat <<EOF | curl -T - -X PUT http://localhost:55555
{"Action":"/usr/bin/Rscript",
"Body":["a <- 1:10","b <- which(a > 2 & a < 8)","b"]}
EOF
\end{lstlisting}

\begin{lstlisting}[language={},name=Rückgabe:]
{
  "t_start":1385566508241,
  "t_stop":1385566508399,
  "exitCode":0,
  "Result":"[1] 3 4 5 6 7\n"
}
\end{lstlisting}

    \item |"/bin/echo"|: \par

      Sendet den Aufrufparameter zurück (für Testzwecke). Weiterer Parameter:
      \begin{description}

        \item[Value] -- Der Aufrufparameter für das Programm.

      \end{description}

    \item |"/usr/bin/which"|: \par

      Testet, ob das als Aufrufparameter angegebene Programm existiert (für
      Testzwecke). Weiterer Parameter:
      \begin{description}

        \item[Value] -- Der Aufrufparameter für das Programm.

      \end{description}

  \end{itemize}

\end{description}

\subsection{Interne Aktionen}

  Es handelt sich um eine vom Server direkt ausführbare Aktion, bei der die
  Angabe des Namens eines externen Programms nicht notwendig ist bzw.\ es
  wird auf ein solches gänzlich verzichtet. Beginnt "`|Action|"' mit dem
  Zeichen "`|_|"', so kennzeichnet sie einen internen Prozess zu
  administrativen Zwecken.

 \begin{description}

  \item[Action] -- kennzeichnet die zu erledigende Aufgabe.

  \begin{itemize}

    \item |"TEX"|: \par

      Bei dieser Aktion handelt es sich eigentlich auch um einen externen
      Programmauf. Aufgrund der Komplexität ist sie dennoch als "`interne
      Aktionen"' eingeordnet.

      Es wird aus dem übergebenen \TeX"=Code eine Datei erzeugt, die mit
      einem der unterstützten \TeX"=Compiler in PDF"=Code (PDF"=Stream)
      konvertiert wird.

      \begin{description}

        \item[Body] -- \TeX"=Code in Form eines Strings oder eines Arrays von
        Strings (zwingend). Im zweiten Falle wird aus dem Array intern ein
        einzelner String erzeugt, wobei nach jedem Inhalt eines
        Array"=Elements ein Zeilenumbruch ("`|\n|"') eingefügt wird.

        \item[Command] -- Der Name des \TeX"=Compilers (optional, Standard:
        "`pdflatex"'). Es werden nur \TeX"=Compiler unterstützt, die auf
        direktem Wege PDF"=Dateien erzeugen, d.\,h.\ "`pdflatex"',
        "`lualatex"', "`xelatex"', "`pdftex"', "`luatex"', "`xetex"'.

        \item[KeepFiles] -- "`true"' oder "`1"' verhindert, dass nach
        erfolgtem Programmaufruf das temporär angelegte Verzeichnis und
        dessen Inhalt gelöschte wird (optional, Standard: "`false"', nur für
        Testzwecke).

      \end{description}

    \item |"TCP"|: \par

    \item |"EMAIL"|: \par

    \item |"RANDOM"|: \par

    \item |"TIME"|: \par

    \item |"_killRepeats"|: \par

    \item |"_version"|: \par

    \item |"_nodesVersion"|: \par

    \item |"_environment"|: \par

  \end{itemize}

\end{description}

\section{Gitlab"=Hook"=Server -- Port: 3420}

Die Web"=Applikation "`GitLab"' gestattet ähnlich zu dem Konkurrenzprodukt
"`GitHub"' einen bequemen Zugriff auf das Versionskontrollsystem "`GIT"'. Zu
jedem  in "`GitLab"' registrierten Repositorium kann über
"`Settings"'/"""`Web hooks"' eine http"=Adresse angegeben werden. Nach
Änderung des betreffende Repositoriums (push"=Aktion) wird an diese Adresse
eine Datenstruktur gesendet. Sie enthält etliche Angaben zu dem
Repositorium, wie Name des Repositoriums, URL für GIT"=Aktionen, Nutzername,
IDs der letzten Commits u.\,v.\,a. "`nodejsServers2"' bietet die
Funktionalität eines Gitlab"=Hook"=Servers und ist somit in der Lage, diese
Informationen von "`GitLab"' zu empfangen und auszuwerten. Die Adresse, die
in "`GitLab"' angegeben werden muss, lautet im Falle des auf "`a73434"'
laufenden "`nodejsServers2"' folgendermaßen:
\begin{lstlisting}[language={}]
  http://a73434.berlin.ptb.de:3420
\end{lstlisting}
Zur Definition dessen, was nach Eintreffen der
Informationen von "`GitLab"' zu tun ist, muss eine Datei |gitlabhook.conf|
angelegt werden. Als Beispiel sei hier der Inhalt angeführt, der dazu führt,
dass sich automatisch mit Änderung des Repositoriums "`vaclabpage"'
html"=Seiten, die vom Webserver ausgeliefert werden, erneuern:
\begin{lstlisting}[language={}]
{
  "tasks": {
    "vaclabpage": [
      "exec 1>/dev/null",
      "exec 2>/dev/null",
      "git clone %h",
      "cd %r",
      "cp -p --parents `git ls-files` /srv/www/htdocs/vaclabpage/"
    ]
  },
  "keep":false
}
\end{lstlisting}
Zur Erklärung: Unter "`tasks"' kann ein oder mehrere Namen von
GIT"=Repositorien aufgeführt werden. Jedem dieser Namen ist ein String oder
ein String"=Array zugeordnet. Darin enthalten sind Unix"=Kommandozeilenaufrufe.
"`|%h|"' ist ein Platzhalter für die zum Clonen des GIT"=Repositorium
verwendbare URL. "`|%r|"' steht für den Namen des Repositoriums. Mit
"`|"keep":false|"' wird festgelegt, dass temporär erstellte Verzeichnisse nicht
erhalten bleiben sollen. Die Beschreibung zu dem NodeJS"=Module
"`node-gitlab-hook"' enthält weitere Hinweise zur Syntax in |gitlabhook.conf|.

Der konkrete Ablauf im temporären Verzeichnis des Rechners
(|/tmp|) zur Erneuerung der Home"=Page des Vakuumlabors ist also der
folgende:
\begin{enumerate}
  \item Lokales Duplikat des GIT"=Repositoriums anlegen.
  \item In das Verzeichnis mit dem Namen des Repositoriums wechseln.
  \item Alle relevanten Dateien zum Webserver"=Verzeichnis kopieren.
\end{enumerate}
Es ist zu beachten, dass alle unter "`nodejsServers2"' laufenden Prozesse
mit den Rechten des Nutzers "`wwwrun"' laufen.

\section{Logging"=Zugriff per Websocket"=Protokoll -- Port: 9001}

Um detaillierte Informationen über den internen Ablauf beim Ansprechen der
unter "`nodejsServers2"' laufenden Server"=Prozesse zu erhalten, kann über
den Port~|9001| per Websocket"=Protokoll Kontakt aufgenommen werden. Zu
diesem Zweck steht das Kommandozeilen"=Programm |vlLogging| zur Verfügung.
Ohne Parameter nimmt es Kontakt zum lokal laufenden "`nodejsServers2"' auf.
Wird als Parameter ein anderer Rechner (Rechnername oder IP"=Adresse)
angegeben, kann auch auf Informationen eines entfernten Rechners zugegriffen
werden. Es wird das folgende Ausgabeformat angeboten:
\begin{lstlisting}[language={}]
2013-10-17 08:53:10.862 - LEVEL: [FILE_NAME:LINE_NUMBER:FUNCTION_NAME] MESSAGE
\end{lstlisting}
\begin{itemize}
  \item |LEVEL|: "`debug"', "`error"', "`warn"' oder "`info"'
  \item |FILE_NAME|: Die Datei, in der sich der abzuarbeitende Code befindet
  \item |LINE_NUMBER|: Zeilennummer in |FILE_NAME|
  \item |FUNCTION_NAME|: Funktion, in der sich der abzuarbeitende Code befindet
  \item |MESSAGE|: Konkrete Informationen zum betreffenden Code"=Teil.
\end{itemize}
Diese Informationen dienen der Fehlersuche im Programmcode von
"`nodejsServers2"' und auch zur allgemeinen Beobachtung z.\,B.\ der
Kommunikation mit Messgeräten. Künftig könnte der Websocket"=Zugriff auch von
einer Web"=Applikation (Web"=Browser) aus erfolgen.

\vfill

\begingroup \small \itshape

\noindent Hinweise zu Verbesserungen dieses Dokuments bitte als
Issue"=Eintrag des git"=Projektes "`nodejsServers2"' (siehe "`GitLab"') oder per
E-Mail an \url{Rolf.Niepraschk@ptb.de}.

\endgroup

\end{document}
%---------------------------
