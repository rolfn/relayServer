%---------------------------
\listfiles
\documentclass[%
fontsize=11pt
,paper=a4
,twoside
,headings=normal
,pagesize
]{scrartcl}
\usepackage{hyphsubst}% Trennregeln austauschen
\HyphSubstIfExists{ngerman-x-latest}{%
  \HyphSubstLet{ngerman}{ngerman-x-latest}}{}

\usepackage[utf8]{inputenc}

\usepackage{cmap}
\usepackage[T1]{fontenc}
\usepackage[largesmallcaps]{kpfonts}
\renewcommand{\sfdefault}{cmbr}
\usepackage[scaled]{luximono}
\usepackage{microtype}
\DeclareMicrotypeAlias{jkp}{ppl}

\usepackage[obeyspaces,spaces]{url}
\usepackage[ngerman]{babel}
\usepackage[autostyle=true,babel=once,german=guillemets,maxlevel=3]{csquotes}%
\defineshorthand{"`}{\openautoquote}
\defineshorthand{"'}{\closeautoquote}

%\setkomafont{minisec}{\normalfont\bfseries}
\setkomafont{disposition}{\normalfont\bfseries}
\setkomafont{descriptionlabel}{\normalfont\bfseries}

\renewcommand*\labelitemii{$\circ$}

\makeatletter
\newcommand\Minisec[1]{\@afterindentfalse \vskip 1.5ex
  {\parindent \z@
    \raggedsection\normalfont\sectfont\nobreak
    \usekomafont{minisec}#1\nobreak}\nobreak%
  \@afterheading
}
\makeatother

\usepackage{shortvrb}
\MakeShortVerb{\|}

\usepackage{xcolor,listings}
\lstset{%
     basicstyle=\footnotesize\ttfamily,
     identifierstyle={},
     keywordstyle=\bfseries,
     %stringstyle=\itshape\color{DTKlstStrings},
     commentstyle=\itshape,
     columns=fixed,
     tabsize=2,
     frame=single,
     extendedchars=true,
     showspaces=false,
     showstringspaces=false,
     breaklines=true,
     breakindent=10pt,
     backgroundcolor=\color{black!10},
     breakautoindent=true,
     captionpos=t,
     aboveskip=\medskipamount,
     belowskip=\medskipamount,
     xrightmargin=\fboxsep,
     prebreak=\mbox{$\hookleftarrow$},
     columns=fullflexible,
     keepspaces=true
}


\setlength\textheight{1.08\textheight}

\shorthandon{"}
\title{Hinweise zur Benutzung des Web"=Services "`NodejsRelay"'}
\date{2011-05-17}
\author{Rolf Niepraschk, AG~7.54}
\shorthandoff{"}

\begin{document}

\maketitle

\section*{Zugriff auf den Web"=Service "`NodejsRelay"'}

Mit Web"=Service "`NodejsRelay"' ist ein spezieller Web"=Service im
Bereich der AG~7.54 gemeint. Auf den beiden zentralen Servern \url
{a73434} und \url{a73435} (ggf. auch auf einigen oder allen
Messrechnern) ist "`NodejsRelay"' installiert. Durch Ansprechen
einer geeigneten http"=Adresse (URL) wird durch "`NodejsRelay"'
jeweils ein Prozess gestartet und dessen Ergebnis zurückgeliefert.
Der Prozess kann eine Messdatenabfrage (z.\,B. \ per
GPIB/LAN"=Adapter), eine Auswerteprozedur (z.\,B.\ mit dem
Statistikprogramm "`R"') oder Anderes sein. Der Zugriff auf
"`NodejsRelay"' geschieht durch Ansprechen einer URL in der Form \url
{http://HOST:55555}. Die Angabe der Portadresse 55555 ist zwingend.

Die auszuführende Aktion wird in der gleichzeitig per |POST| oder
|PUT| mitgeschickten JSON"=Datenstruktur (siehe nächster Abschnitt)
angegeben. Ein testweiser Aufruf mit dem Programm "`curl"' könnte
z.\,B.\ folgendermaßen aussehen:

\medskip

  \qquad \url+echo '{"Action":"RANDOM"}' | \ +
  \par\vspace{0\baselineskip}
  \qquad\quad \url+curl -T - -X POST http://a73434.berlin.ptb.de:55555+
  
\medskip

\noindent Die Rückgabe erfolgt ebenfalls in Form einer
JSON"=Datenstruktur und würde in diesem Fall ähnlich wie hier
aussehen:

\medskip

\qquad |{"Result":[0.6832635675091296]}|



\section*{Datenstruktur zur Steuerung des Web"=Services "`NodejsRelay"'}

Zur Steuerung des Web"=Services "`NodejsRelay"' wird die folgende
Datenstruktur (JSON) verwendet (Beispiel zur Temperaturmessung
mit dem Multimeter "`Agilent 34970A"'):

\begin{lstlisting}[language=Java]
{
  "Action": "/usr/local/bin/vxiTransceiver",
  "Host":"e75481",
  "Device":"gpib0,5",
  "Value":"READ?",
  "PostProcessing": [
    "_tmp=_x.split(',');",
    "_tmp.forEach(function(e,i,a){a[i]=parseFloat(e)});",
    "T101=_tmp[0]",
    "T102=_tmp[1]",
    "T108=_tmp[2]"
  ],
  "DemoMode": false,
  "DemoResponse": "23.1,23.2,23.3",
  "Comment": "VXI-Kommunikation: T-Messung"
}
\end{lstlisting}

\noindent Zusätzliche Einträge werden ignoriert. Mindestens muss der Eintrag
"`|Action|"' vorhanden sein.

\subsection*{Erläuterungen zu allgemeinen Eintragstypen:}

\begin{description}

  \item[Action] -- kennzeichnet den Prozess, der von "`NodejsRelay"'
  gestartet wird. Es werden zwei Arten unterschieden:
  
  \begin{itemize}

    \item Externe Prozesse: "`|Action|"' ist der absolute Pfad zu
    einem ausführbaren Programm, welches auf dem "`NodejsRelay"'"=Server
    gestartet werden soll. Der angegebene String muss mit dem Zeichen
    "`|/|"' beginnen. Es sind nur
    ausgewählte Programme erlaubt. Derzeit sind dies:
    \begin{itemize}

    \item |/usr/local/bin/vxiTransceiver| \par
    
      Parameter~1 gibt den Hostnamen oder die IP"=Adresse und
      Parameter~2 den Namen des internen devices eines
      VXI\,11"=kompatiblem Netwerkgerätes an. Ist Parameter~3 eine
      Zahl, so gibt sie an, wie lange maximal auf Rückgabedaten
      gewartet wird. Im anderen Fall ist es der erste Teil der zu
      sendenden Zeichenkette, wobei dann 2000\,ms als Wartezeit
      angenommen wird. Die explizite Angabe einer Wartezeit von 0
      kann benutzt werden, um z.\,B. \ bei fehlerhaften
      GPIB-Geräten einen "`timeout error"' zu vermeiden, wenn diese
      zwar Daten aber kein korrektes Ende"=Signal senden.
      
    \item |/usr/bin/R|
    \item |/bin/echo|

    \end{itemize}

    \item Interne Prozesse: "`|Action|"' enthält nicht das Zeichen
    "`|/|"'. Es handelt sich um eine vom Server direkt ausführbare
    Aktion, bei der die Angabe des Namens eines externen Programms
    nicht notwendig ist bzw.\ es wird auf ein solches gänzlich
    verzichtet. Beginnt "`|Action|"' mit
    dem Zeichen "`|_|"', so kennzeichnet sie einen internen Prozess
    zu administrativen Zwecken. Zur Zeit werden folgende interne
    Prozesse unterstützt:

    \begin{itemize}
    
    \item |TCP| -- Baut eine Netzwerkverbindung zu dem angegebenen
    Rechner (Parameter |Host|) mit der angegebenen Portnummer
    (Parameter |Port|) auf und sendet die als Parameter |Value|
    angebene Zeichenkette.

    \item |EMAIL| -- Versendet eine E"=Mail an einen oder mehrere
    Adressaten. Näheres siehe weiter unten.
    
    \item |RANDOM| -- Liefert zu Testzwecken eine oder mehrere
    Zufallszahlen.

    \item |_version| -- Sendet die Versionsnummer des laufenden
    NodejsRelay"=Services. Derzeit ist die Version |2.3j| aktuell.

    \item |LaTeX| -- Liefert pdf"=Code uncodiert oder
    base64"=codiert aufgrund von als Parameter übergebenem
    \LaTeX"=Quellcode. Näheres siehe weiter unten.
    
    \end{itemize}
    
  \end{itemize}

  Die Angabe von "`|Action|"' ist zwingend.

  \item[Host] -- Gibt den Rechner, der zur Kommunikation benutzt werden
  soll, in Form seiner IP-Adresse oder seines Namens an. (optional)

  \item[Device] -- Im Falle von |vxiTransceiver| kennzeichnet diese
  Angabe das interne VXI11-Gerät. (optional)

  \item[Port] -- Im Falle von |TCP| kennzeichnet diese
  Angabe die für die Kommunikation zu verwendende Port"=Nummer. (optional)

  \item[PostProcessing] -- Bestimmt, welche Variablen mit welchen
  Werten in der Rückgabestruktur erscheinen sollen. Es existiert
  implizit eine Variable "`|_x|"', die die unbehandelten
  Rückgabedaten des ausgeführten Prozesses enthält. Diese und alle
  anderen Variablen, die mit "`|_|"' beginnen, werden als temporär
  betrachtet und erscheinen nicht in der Rückgabestruktur. Es können
  alle Sprachkonstrukte, die
  der Javascript"=Interpreter "`V8"' versteht, genutzt werden können.
  Ohne Angabe von |PostProcessing| ist die Rückgabe \par\medskip
  \quad |{"Result":"???"}| \par\medskip
  |???| symbolisiert die unbehandelten Rückgabedaten. (optional)

  \item[DemoMode] -- Gibt an, ob Werte des Prozesses oder die in
  "`|DemoResponse|"' angegebenen zurückgeliefert werden sollen
  (optional, |true| oder |false|).

  \item[DemoResponse] -- Rückgabewert im Falle von
  "`|DemoMode:true|"'. Es ist sinnvoll, wenn "`|DemoResponse|"'
  strukturell mit dem regulären Rückgabewert des Prozesses
  übereinstimmt, da auch "`|DemoResponse|"' per "`|PostProcessing|"'
  behandelt wird. Die aktivierte Rückgabe von "`|DemoResponse|"' ist
  zur Validierung der gesamten Prozesskette oder für andere
  Testzwecke sinnvoll. (optional)

  \item[OutputType] -- Kennzeichnet, wie die Rückgabedaten gesendet
  werden. Im Falle von |"OutputType":"stream"| handelt es sich um
  unstrukturierte Daten ohne jegliches "`PostProcessing"'. Im
  Standardfall (|"OutputType":"json"|) wird eine
  JSON"=Datenstruktur, die ggf.\ per "`PostProcessing"' bearbeitet
  wurde, zurückgeschickt. (optional)

  \item[OutputEncoding] -- Kennzeichnet, in welcher Weise die
  Rückgabedaten codiert sind. Standardwert ist |"utf8"|, was im
  Falle von |"OutputType":"json"| erzwungen wird. Als Alternativen
  existieren |"base64"| und |"binary"|, wobei letztere Angabe nur
  intern Verwendung findet. (optional)

  \item[Comment] -- Wird nicht ausgewertet, sollte aber dennoch
  enthalten sein, um die Eigenschaften zu beschreiben.

\end{description}

\subsection*{Erläuterungen zu Eintragstypen für den Action"=Typ
"`\texttt{EMAIL}"':}

\begin{description}

  \item[Host] -- SMPT"=Server, der die Weiterleitung der E-Mail
  bewirkt (Standard: |"smtp-hub"|, optional).

  \item[Port] -- Portadresse des SMPT"=Servers (Standard: |25|).

  \item[Subject] -- (optional).

  \item[From] -- Absender (zwingend).

  \item[To] -- Eine oder mehrere kommaseparierte E-Mail"=Adressen
  (zwingend).

  \item[Cc, Bcc] -- Eine oder mehrere kommaseparierte
  E-Mail"=Adressen (optional).

  \item[Body] -- Zu sendender einfacher Text (optional).

  \item[Html] -- Zu sendender Text im HTML"=Format (optional).

  \item[Attachments] -- Ein Array von Objekten, von denen jedes eine
  Datei kennzeichnet (optional). Ein solches Objekt verfügt über die
  folgenden Eintragstypen:

  \begin{description}

    \item[Filename] -- Der Name der Datei (zwingend). 

    \item[Contents] -- Der Dateiinhalt in base64"=Codierung (zwingend).
    
  \end{description}

\end{description}

\subsection*{Erläuterungen zu Eintragstypen für den Action"=Typ
"`\texttt{LaTeX}"':}

\begin{description}

  \item[SrcFormat] -- Legt fest, ob es sich um \LaTeX"=Quelltext
  (|"LaTeX"|) oder plain\TeX"=Quelltext (|"TeX"|) handelt. Ohne
  Angabe wird |"LaTeX"| angenommen. (optional)

  \item[DestFormat] -- Legt fest, ob die Rückgabe im pdf"=Format
  (|"PDF"|) oder dvi"=Format (|"DVI"|) erfolgen soll. Ohne
  Angabe wird |"PDF"| angenommen. (optional)

  \item[Source] -- Der Quell"=Code des \LaTeX"=Dokumentes als String
  oder als String"=Array. Der String kann base64"=codiert oder
  uncodiert angegeben werden, wobei "`NodejsRelay"' eine vorhandene
  Codierung selbsttätig erkennt. Im letzteren Fall ist darauf zu
  achten, dass das Zeichen "`|\|"' doppelt angegeben wird, um zu
  verhindern, dass es vom Javascript"=Interpreter als Sonderzeichen
  interpretiert wird. Ohne Angabe wird die erste unter
  |"Attachments"| angegebene Datei als \LaTeX"=Dokument verwendet.
  (optional)

  \item[MaxRuns] -- Anzahl der maximal durchgeführten \LaTeX"=Läufe
  bis angenommen wird, dass das Ergebnis stabil ist. Ohne Angabe
  wird "`2"' angenommen. (optional)

  \item[KeepFiles] -- Legt fest, ob das temporäre Verzeichnis
  einschließlich seines Inhaltes am Ende erhalten bleibt oder nicht.
  Standardwert ist |"false"|. Dieser Eintrag hat nur für Testzwecke
  eine Bedeutung. (optional)

  \item[Attachments] -- Ein Array von Objekten, von denen jedes eine
  externe Datei zur Verwendung beim \LaTeX"=Lauf kennzeichnet
  (optional). Ein solches Objekt verfügt über die folgenden
  Eintragstypen:

  \begin{description}

    \item[Filename] -- Der Name, unter dem die Datei erzeugt werden
    soll. Es gelten die unter unix"=artigen Betriebssystemen gültigen
    Einschränkungen.

    \item[Contents] -- Der Dateiinhalt. Bezüglich der
    Codierung gilt das zu |"Source"| Gesagte. Im Falle von
    Binärdateien (z.\,B.\ jpeg"=Grafik) ist base64"=Codierung zwingend.
    
  \end{description}

\end{description}

\noindent Speziell beim Action"=Typ "`\texttt{LaTeX}"' kann die Zuweisung
|"stream"| an den allgemeinen Parameter |"OutputType"| sinnvoll
sein. In diesem Falle werden die reinen pdf"=Daten gesendet.

\section*{Zu "`NodejsRelay"' gehörige Dateien}

"`NodejsRelay"' erfordert, dass folgende Einzeldateien und
Softwarepakete installiert sind (Stand: 2011-05-13):

\begin{enumerate}

  \item Das Paket "`|nodejs3|"':\par%\vspace{-\baselineskip}
\begin{lstlisting}
zypper ar \
http://download.opensuse.org/repositories/home:/maw:/nodejs/openSUSE_11.4/home:maw:nodejs.repo
zypper ref
zypper in nodejs
\end{lstlisting}

  \item Das Javascript"=Programm mit der Server"=Funktionalität:
  \par\vspace{-.25\baselineskip}
  \path{/usr/local/bin/relay.js}

  \item Das Startscript \path{/usr/local/bin/nodejsRelay}

  \item Das  System"=Startscript \path{/etc/init.d/NodejsRelay} mit
  symbolischem Link nach \path{/usr/sbin/rcNodejsRelay}

  \item Script und Binärprogramm zur VXI\,11"=Kommunikation:
  \par\vspace{-.25\baselineskip}
  \path{/usr/local/bin/vxiTransceiver} und
  \path{/usr/local/bin/vxi11_transceiver}

\end{enumerate}

\noindent Mit "`yast"' kann erreicht werden, dass der Service
"`NodejsRelay"' in den Run"=Leveln "`3"' und "`5"' automatisch
gestartet wird.

\end{document}
%---------------------------
Beispiele:

echo '{"Action":"TCP","Params":["a73434",5984,"GET / HTTP/1.0\n\n"]}'| \
  curl -T - -X PUT http://i75434.berlin.ptb.de:55555

echo '{"Action":"RANDOM"}' | \
  curl -T - -X PUT http://a73434.berlin.ptb.de:55555
